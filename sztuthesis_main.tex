\documentclass{SZTUthesis}
\usepackage{makecell}
\usepackage{algorithm}
\usepackage{algorithmicx}
\usepackage{setspace}
\setstretch{1.5}
\hypersetup{
colorlinks=true,
linkcolor=black
}

\newcommand\pkg[1]{\texttt{#1}\textsuperscript{\sffamily PKG}}
\newcommand\env[1]{\texttt{#1}\textsuperscript{\sffamily ENV}}
\newcommand\app[1]{\textsf{#1}}
\newcommand\oper[1]{\texttt{#1}}
\newcommand\cls[1]{\texttt{#1}\textsuperscript{\sffamily CLS}}
\newcommand\bib[1]{\texttt{#1}\textsuperscript{\sffamily BIB}}
\renewcommand\emph[1]{\textbf{#1}}
\newcommand\format[1]{\textsf{#1}}

%算法
%\begin{algorithm}[t]
%\caption{algorithm caption} %算法的名字
%\hspace*{0.02in} {\bf Input:} %算法的输入, \hspace*{0.02in}用来控制位置,同时利用 \\ 进行换行
%input parameters A, B, C\\
%\hspace*{0.02in} {\bf Output:} %算法的结果输出
%output result
%\begin{algorithmic}[1]
%\State some description % \State 后写一般语句
%\For{condition} % For 语句,需要和EndFor对应
%	\State ...
%	\If{condition} % If 语句,需要和EndIf对应
%		\State ...
%	\Else
%		\State ...
%	\EndIf
%\EndFor
%\While{condition} % While语句,需要和EndWhile对应
%	\State ...
%\EndWhile
%\State \Return result
%\end{algorithmic}
%\end{algorithm}

%!TEX root = ../sztuthesis_main.tex
% 文章信息
\titlecn{深圳技术大学本科学位论文LaTeX模板}
\titleen{LaTeX Template of Postgraduate Thesis of \\Shenzhen Technology University}

\priormajor{物联网工程}
\author{郭大侠}
\supervisor{我自己}
\supervisortitle{助理教授}
\department{大数据与互联网学院}
\studentid{123456789}
\thesisdate{year=2021,month=05,day=01}



\clcnumber{TP391} 				% 中图分类号 Chinese Library Classification
\schoolcode{14655}			% 学校代码
\udc{004.9}						% UDC
\academiccategory{学术学位}	% 学术类别


\newif \ifblindreview % 条件语句,是否是盲审版本
% \blindreviewtrue
\blindreviewfalse

% lipsum
\newcommand{\lipsum}{%
这是一段随机插入的文本,用来填充模板布局,感受模板视觉效果。

深圳技术大学是广东省和深圳市高起点、高水平、高标准建设的本科层次公办普通高等学校。2015年,深圳市委市政府开始筹建深圳技术大学。2016年3月,深圳市人民政府办公厅发布关于设立深圳技术大学筹备办公室的通知。2017年7月,深圳市机构编制委员会发布关于设立深圳技术大学(筹)的通知。2017年9月、2018年9月深圳技术大学(筹)依托深圳大学分别招收了226人和807人。2018年11月30日,经教育部批准正式设立深圳技术大学,学校独立招生,标识码为4144014655,定位于应用型高等学校。2019年9月,学校首年独立招生录取807人,招生的六个省份均高于一本线(高优线/自招线)录取;其中,广东省理科投档线进入前十。

学校充分借鉴和引进德国、瑞士等发达国家一流技术大学先进的办学经验,致力于培养本科及以上层次具有国际视野、工匠精神和创新创业能力的高水平工程师、设计师等高素质应用型人才,努力建成一流的应用型技术大学。

着力建设面向国家和地方发展需要的,以工学为主,理学、管理学、艺术学等协调发展的学科体系,并按计划分布发展和优化学科布局。

目前设立了中德智能制造学院、大数据与互联网学院、新材料与新能源学院、城市交通与物流学院、健康与环境工程学院、创意设计学院、工程物理学院、质量和标准学院、国际交流学院、商学院、药学院、外国语学院、马克思主义学院(人文社科学院)、体育学院。已开设机械设计制造及其自动化、物联网工程、光源与照明、交通运输、汽车服务工程、工业设计等高度契合经济发展和产业需求的专业。至2022年,学校拟开设专业39个,涵盖工学、理学、管理学、艺术学、经济学等5个学科门类。

这是一段随机插入的文本,用来填充模板布局,感受模板视觉效果。%
}

% 让各类元素在全文按1、2、3...顺序编号,而不因章节变化而重置为1
\numberwithin{figure}{section}
\numberwithin{equation}{section}
\numberwithin{algorithm}{section}
\numberwithin{table}{section}

\begin{document}
%%%%%%%%%%%%%%%%%%%%%%%%%%%%%%%%%%%%%%%%%%%%%%%%%%
% 封面
% -----------------------------------------------%
\makecoverpage

%%%%%%%%%%%%%%%%%%%%%%%%%%%%%%%%%%%%%%%%%%%%%%%%%%
% 前置部分的页眉页脚设置
% -----------------------------------------------%
\newpage


\pagestyle{empty}
%%%%%%%%%%%%%%%%%%%%%%%%%%%%%%%%%%%%%%%%%%%%%%%%%%
% 声明页
% -----------------------------------------------%
\announcement
\newpage


% 目录
% -------------------------------------------%
{
\renewcommand{\contentsname}{\hfill \heiti \zihao{-2} 目\quad 录\hfill}  
	\zihao{4}
	\setlength{\parskip}{6pt}  			% 段间距,补偿latex行间距计算(与word的差距)
	\renewcommand*{\baselinestretch}{1.5}   % 行间距
    \tableofcontents
}
\newpage

% 去掉页眉章节序号后面的“.”
\renewcommand{\sectionmark}[1]{\markright{\thesection~ #1}} 
\renewcommand{\headrulewidth}{1pt}



% 正文内容 
% --------------------------------------------%

\setheader

% 可以使用include命令导入tex文件,从而避免过多修改本文件。

% 论文正文是主体,主体部分应从另页右页开始,每一章应另起页。一般由序号标题、文字叙述、图、表格和公式等五个部分构成。

% 重新设置正文行间距,因为前置部分设置时候行间距被改过
\renewcommand*{\baselinestretch}{1.5}   % 几倍行间距
\setlength{\baselineskip}{16pt}         % 基准行间距

% 正文
{
% 表格字号应比正文小,这里设为五号,但是暂时没法在cls里设置(不然会影响到封面等tabular环境)
% 所以目前只好在主文件里局部\AtBeginEnvironment
	\AtBeginEnvironment{tabular}{\zihao{5}}
	\infotitle	% 正文标题,带学生信息
	\linespread{1.5}
	% 中文摘要
	\setabstractfooter
	
	\input{content/abstractcn}
	\newpage

	%%%%%%%%%%%%%%%%%%%%%%%%%%%%%%%%%%%%%%%%%%%%%%%%%%
	% 英文摘要
	% -----------------------------------------------%
	\include{content/abstracten}

	% 正文内容
	\setfooter
	 
	\begin{spacing}{1.5}
		%!TEX root = ../sztuthesis_main.tex
% 论文正文是主体,主体部分应从另页右页开始,每一章应另起页。一般由序号标题、文字叙述、图、表格和公式等五个部分构成。
\section{引言}
\subsection{研究背景与意义}

\app{Word} 不难用,但是想用得漂亮还得费一番功夫。

对于小白来说,(注意!!是对于小白来说,不要跟我杠!!!我就是 \app{Word} 小白,高端玩法玩不动):

插入个图片,下面的说明文字是不是插入文本框?那文本框要不要跟图片“组合”?是不是直接圈没法圈起来?因为图片要变成浮动格式,和文本框绑定后再改回嵌入格式,你说蛋疼不蛋疼?不组合?那有一定概率发生你的图片在上一页,描述文字在下一页。呵呵。

插入参考文献,手动编辑?我的天哪,一百多个文献,中间插一个,怎么改序号?
很好,可以用交叉引用,一个个编辑文献格式?
很好,可以用 \app{endnote} 或者 \app{noteexpress} 的插入功能,你有没有发现插入是个宏,\oper{ctrl+z} 的时候会烦死你啊……

叮叮!让我们祭出 \LaTeX!!\cite{2013浅谈使用},有 \env{bibliography},一个 \pkg{cite} 包打天下!不要太爽。


\begin{figure}[hbt]
    \centering
    \includegraphics[width=0.5\textwidth]{latex.jpg}
    \caption{LaTeX}
    \label{F.latex}
\end{figure}

插入公式,对 \app{Word} 小白来说,公式居中编号靠右就是一道百度搜索能力过滤器。

\app{Word} 里编辑三线表,啊烦躁。

等等等等……

让我们,专心写论文好不好?

爱你们。


\subsection{主要研究工作}
虽然我 \LaTeX 水平也很水……但是通过大量 \oper{debug} 也勉强给大家凑出来一个格式绝对标准的 \LaTeX 模板,模板代码丑就丑吧,能用就行。写了大量注释,有一点 \LaTeX 基础就可以根据自己需要修改 \cls{SZTUthesis.cls} 文件。

(1) 提供图片插入示例。

(2) 提供表格插入示例。

(3) 提供公式插入示例。

(4) 提供参考文献插入示例。

\subsection{论文组织结构}

全文内容共六章,具体内容组织如下:

第一章为绪论。

第二章为图片插入示例。

第三章为表格插入示例。

第四章为公式插入示例。

第五章为参考文献插入示例。

第六章总结与展望,总结了本文的主要工作,展望了下一阶段的研究方向。

% \newpage    % 两个章节之间分页,不想分的话可注释掉

\section{图像布局}
\label{sec.figure}

\emph{学校对图片只有小标题要求,没有进一步的子图要求,我们按科技论文常规排版来}

\subsection{单图布局}

\lipsum

\emph{单图布局如图~\ref{F.sztu_single} 所示。}

\begin{figure}[hbt]
\centering
\includegraphics[width=0.5\textwidth]{sztu.png}
\caption{单图布局示例}
\label{F.sztu_single}
\end{figure}

\subsection{单图位置固定布局示例}
在LaTeX中,figure环境用于插入和管理图片或图表,而放置图片的方式由可选参数来控制。H是一个浮动参数选项,表示强制将图片放置在其出现的位置。具体来说,H是由float包提供的选项,如果要使用它,需要在导言区加载float包:
\begin{figure}[H]
    \centering
    \includegraphics[width=0.5\linewidth]{sztu.png}
    \caption{图片位置固定布局示例}
    \label{fig:H_example}
\end{figure}
这样做会强制LaTeX将图片放在代码中的确切位置,而不会尝试将其移动到其他页面或位置。这个选项对于需要严格控制图片位置的文档特别有用。在这个示例中,figure环境中的图片会在其代码位置处被强制放置,而不会被浮动到文档的其他地方。

\subsection{横排布局}

\emph{横排布局如图~\ref{F.sztu_row} 所示。}

\begin{figure}[!htb]
    \centering
    \begin{subfigure}[t]{0.24\linewidth}
        \begin{minipage}[b]{1\linewidth}
        \includegraphics[width=1\linewidth]{sztu.png}
        \caption{可以增加描述}
        \end{minipage}
    \end{subfigure}
    \begin{subfigure}[t]{0.24\linewidth}
        \begin{minipage}[b]{1\linewidth}
        \includegraphics[width=1\linewidth]{sztu.png}
        \caption{}
        \end{minipage}
    \end{subfigure}
    \begin{subfigure}[t]{0.24\linewidth}
        \begin{minipage}[b]{1\linewidth}
        \includegraphics[width=1\linewidth]{sztu.png}
        \caption{}
        \end{minipage}
    \end{subfigure}
    \begin{subfigure}[t]{0.24\linewidth}
        \begin{minipage}[b]{1\linewidth}
        \includegraphics[width=1\linewidth]{sztu.png}
        \caption{}
        \end{minipage}
    \end{subfigure}
    \caption{横排布局示例}
    \label{F.sztu_row}
\end{figure}

\lipsum

\subsection{竖排布局}
\emph{竖排布局如图\ref{F.sztu_col}所示。}

\begin{figure}[!htb]
    \centering
    \begin{subfigure}[t]{0.15\linewidth}
        \begin{minipage}[b]{1\linewidth}
        \includegraphics[width=1\linewidth]{sztu.png}
        \caption{}
        \end{minipage}
    \end{subfigure}\\
    \begin{subfigure}[t]{0.15\linewidth}
        \begin{minipage}[b]{1\linewidth}
        \includegraphics[width=1\linewidth]{sztu.png}
        \caption{}
        \end{minipage}
    \end{subfigure}
    \caption{竖排布局示例}
    \label{F.sztu_col}
\end{figure}

\lipsum

\subsection{竖排多图横排布局}

\begin{figure}[!htb]
    \centering
    \begin{subfigure}[t]{0.13\linewidth}
        \begin{minipage}[b]{1\linewidth}
        \includegraphics[width=1\linewidth]{sztu.png} \vspace{-1ex} \vfill
        \includegraphics[width=1\linewidth]{sztu.png}
        \end{minipage}
        \caption{}
    \end{subfigure}
    \begin{subfigure}[t]{0.13\linewidth}
        \begin{minipage}[b]{1\linewidth}
        \includegraphics[width=1\linewidth]{sztu.png} \vspace{-1ex} \vfill
        \includegraphics[width=1\linewidth]{sztu.png}
        \end{minipage}
        \caption{}
    \end{subfigure}
    \caption{竖排多图横排布局}
    \label{F.sztu_col_row}
\end{figure}

\emph{竖排多图横排布局如图~\ref{F.sztu_col_row} 所示。注意看(a)、(b)编号与图关系。}


\subsection{横排多图竖排布局}

\lipsum

\begin{figure}[!htb]
    \centering
    \begin{subfigure}[t]{0.3\linewidth}
        \begin{minipage}[b]{1\linewidth}
        \includegraphics[width=0.45\linewidth]{sztu.png}
        \includegraphics[width=0.45\linewidth]{sztu.png}
        \end{minipage}
        \caption{}
    \end{subfigure}\\
    \begin{subfigure}[t]{0.3\linewidth}
        \begin{minipage}[b]{1\linewidth}
        \includegraphics[width=0.45\linewidth]{sztu.png}
        \includegraphics[width=0.45\linewidth]{sztu.png}
        \end{minipage}
        \caption{}
    \end{subfigure}
    \caption{横排多图竖排布局}
    \label{F.sztu_row_col}
\end{figure}

\emph{横排多图竖排布局如图~\ref{F.sztu_row_col} 所示。注意看(a)、(b)编号与图关系。}

\subsection{本章小结}
本章示例图片布局。

这里再测试一下不同章节的公式编号
\begin{equation}
p_{i} = \frac{e^{-\varepsilon_{i}/kT}}{\sum_{j=1}^{M}e^{-\varepsilon_{j}/kT}}
\end{equation}

\newpage    % 两个章节之间分页,不想分的话可注释掉


\section{表格插入示例}

\begin{table}[htb]
  \centering
  \caption{学校文件里对表格的要求不是很高,不过按照学术论文的一般规范,表格为三线表。}
  \label{T.example}
  \begin{tabular}{llllll}
  \hline
   & A  & B  & C  & D  & E \\
  \hline
1 	& 212 & 414 & 4 		& 23 & fgw	\\
2 	& 212 & 414 & v 		& 23 & fgw	\\
3 	& 212 & 414 & vfwe		& 23 & 长一些的内容	\\
4 	& 212 & 414 & 4fwe		& 23 & 嗯	\\
5 	& af2 & 4vx & 4 		& 23 & 长一些的内容	\\
6 	& af2 & 4vx & 4 		& 23 & fgw	\\
7 	& 212 & 414 & 4 		& 23 & fgw	\\

\hline{}
\end{tabular}
\end{table}

\emph{表格如表~\ref{T.example} 所示,\LaTeX 表格技巧很多,这里不再详细介绍。}

\lipsum

\newpage    % 两个章节之间分页,不想分的话可注释掉

\section{公式插入示例}

\lipsum

\emph{公式插入示例如公式~\eqref{E.example} 所示。}
\begin{equation}
\gamma_x=
\begin{cases}
  0, & \text{if $|x| \leq \delta$} \\
  x, & \text{otherwise}
\end{cases}
\label{E.example}
\end{equation}


\newpage    % 两个章节之间分页,不想分的话可注释掉

\section{参考文献插入示例}

\LaTeX \cite{lamport1994latex}插入参考文献最方便的方式是使用 \env{bibliography}\cite{pritchard1969statistical}。

大多数出版商的论文页面都会有导出 \format{bib} 格式参考文献的链接,把每个文献的 \format{bib} 放入 \bib{thesis-references.bib},然后用 \oper{bibkey} 即可插入参考文献。

\lipsum

\newpage    % 两个章节之间分页,不想分的话可注释掉


\section{总结与展望}

\noindent{纯数字编号}
\begin{enumerate}
 \item XXXXXXXXXX
 \label{item1}
 \item XXXXXXXXXX
 \item XXXXXXXXXX
\end{enumerate}
罗马编号
\begin{enumerate}[label=(\roman*)]
 \item XXXXXXXXXX
 \label{item2}
 \item XXXXXXXXXX
 \item XXXXXXXXXX
\end{enumerate}
括号编号
\begin{enumerate}[label=(\arabic*)]
 \item XXXXXXXXXX
 \label{item3}
 \item XXXXXXXXXX
 \item XXXXXXXXXX
\end{enumerate}
半括号编号
\begin{enumerate}[label=\arabic*)]
 \item XXXXXXXXXX
 \label{item4}
 \item XXXXXXXXXX
 \item XXXXXXXXXX
\end{enumerate}
小字母编号
\begin{enumerate}[label=\alph*)]
 \item XXXXXXXXXX
 \label{item5}
 \item XXXXXXXXXX
 \item XXXXXXXXXX
\end{enumerate}

引用测试,正如~\ref{item1}、\ref{item2}、\ref{item3}、\ref{item4}、\ref{item5} 所示

\subsection{工作展望}
手动编号 %(不推荐,无法被交叉引用)

本课题针对XX,鉴于XXX,对XX进行了提高,但是XXX,所以有如下XX:

(1)目前XX虽然XX,但是XX仍然XX,所以XX仍然是一个值得XX的问题。

(2)随着XX,XX具有XX的问题,仍值得进一步XX。

(3)本课题在XX有了XX,但是XX的XX还存在XX,所以XX。


\newpage

	\end{spacing}
	

}

%%%%%%%%%%%%%%%%%%%%%%%%%%%%%%%%%%%%%%%%%%%%%%%%%%
% 临时标签,用于编译时追踪正文末尾
%%%%%%%%%%%%%%%%%%%%%%%%%%%%%%%%%%%%%%%%%%%%%%%%%%

%%%%%%%%%%%%%%%%%%%%%%%%%%%%%%%%%%%%%%%%%%%%%%%%%%
% 后续内容
% --------------------------------------------%

% https://www.zhihu.com/question/29413517/answer/44358389 %
% 说明如下:
% secnumdepth 这个计数器是 LaTeX 标准文档类用来控制章节编号深度的。在 article 中,这个计数器的值默认是 3,对应的章节命令是 \subsubsection。也就是说,默认情况下,article 将会对 \subsubsection 及其之上的所有章节标题进行编号,也就是 \part, \section, \subsection, \subsubsection。LaTeX 标准文档类中,最大的标题是 \part。它在 book 和 report 类中的层级是「-1」,在 article 类中的层级是「0」。这里,我们在调用 \appendix 的时候将计数器设置为 -2,因此所有的章节命令都不会编号了。不过,一般还是会保留 \part 的编号的。所以在实际使用中,将它设置为 0 就可以了。

% 在修改过程中请注意不要破环命令的完整性

\renewcommand\appendix{\setcounter{secnumdepth}{-2}}
\appendix

% 主文件有代码去掉页眉章节编号的“.”,但这会因为bug导致无编号章节显示一个错误编号,所以这里在无编号章节之前再次重定义sectionmark。
\renewcommand{\sectionmark}[1]{\markright{#1}}

\setreference

%!TEX root = ../sztuthesis_main.tex

% 致谢
\begin{thankscontent}
    本文能顺利完成,首先最要感谢的是...
    
    为了能把主要精力放在论文撰写上,许多国际期刊和高校都支持LaTeX的撰写与提交,新手不需要关心格式问题,只需要按部就班的使用少数符号标签,即可得到符合要求的文档。且在需要全篇格式修改时,更换或修改模板文件,即可直接重新编译为新的样式文档,这对于word新手使用word的感受来说是不可思议的。
    
    本项目的目的是为了创建一个符合深圳技术大学学位论文撰写规范的TeX模板,解决学位论文撰写时格式调整的痛点。
    
    % 图X幅,表X个,参考文献X篇(四号宋体)
    
\end{thankscontent}
 


\end{document}
