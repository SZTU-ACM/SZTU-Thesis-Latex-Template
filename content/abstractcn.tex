%!TEX root = ../sztuthesis_main.tex
% 设置中文摘要

\phantomsection
\addcontentsline{toc}{section}{摘要}

\keywordscn{深圳技术大学;学位论文;LaTeX模板}
\categorycn{TP391}
\begin{abstractcn}
LaTeX利用设置好的模板,可以编译为格式统一的pdf。目前国内大多出版社与高校仍在使用word,word由于其强大的功能与灵活性,在新手面对形式固定的论文时,排版、编号、参考文献等简单事务反而会带来很多困难与麻烦,对于一些需要通篇修改的问题,要想达到LaTeX的效率,对word使用者来说需要具有较高的技能水平。

为了能把主要精力放在论文撰写上,许多国际期刊和高校都支持LaTeX的撰写与提交,新手不需要关心格式问题,只需要按部就班的使用少数符号标签,即可得到符合要求的文档。且在需要全篇格式修改时,更换或修改模板文件,即可直接重新编译为新的样式文档,这对于word新手使用word的感受来说是不可思议的。

本项目的目的是为了创建一个符合深圳技术大学学位论文撰写规范的TeX模板,解决学位论文撰写时格式调整的痛点。

% 图X幅,表X个,参考文献X篇(四号宋体)

\end{abstractcn}